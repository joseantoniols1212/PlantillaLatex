\section{Poligonales de Euler de soluciones \( \epsilon \) -aproximadas.}

Supongamos que:
\[
  (P)
  \begin{cases}
    x' = f(t,x) \\
    x(t_0) = x^0
  \end{cases}
\]
siendo \( f : D \subset \mathbb{R}^{n} \to \mathbb{R}^{n}, (t_0,x^0) \in D \) .
Tiene solución \( f : I \to \mathbb{R}^{n} \)

Una parametrización de la grafica de \( \varphi \) (contenida en D) viene dada por:
\[
  \gamma : t \in I \mapsto (t, \varphi(t)) \in \mathbb{R}^{n+1}
\]

El vecor tangente a esta grafica viene dada por \( (1,\varphi(t))\in \mathbb{R}^{n+1} \)

La recta tangente a la grafica de \( \varphi \) en el punto \( (t,\varphi(t)) \in D \) tiene la siguiente ecuación paramétrica:
\[
  \begin{cases}
    t = t_1 + \lambda \\
    x = \varphi(t_1) + \lambda \varphi (t_1)
  \end{cases}
\]

Sin parámetros:
\[
  x = \varphi (t_1) + (t+t_1) \varphi'(t_1) =  \varphi (t_1) + (t+t_1)f(t_1, \varphi(t_1)) 
\]

La trecta tangente es la mejor aproximacion lineal de \( \varphi \) en un entorno de \( t_1 \).
Una posible aproximacion de \( \varphi \) en un intervalo de la forma \( I_{n}^+ = [t_0, t_0 + h] \) podria darse usando trozos de rectas tangentes.
Tomamamos una particion de dicho intervalo \( I_{n}^+ = [t_0, t_0 + h] \) : 
\[
  \Pi = \{ t_0 < t_1 < ... < t_m = t_0+h \}
\]
Partiendo de \( (t_0, x^0) \) trazamos el segmento a la drecha de \( t_0 \) con vector direcotr \( (1, f(t_0, x^0)) \) hasta llegar al punto \( (t_0, x^0) \) (que seguramente ya no pertence a la gráfica de \( \varphi \) pero queda próximo).
Siguiente trozo: partiendo de \( (t,x') \) trazamos otro segmento a la drecha de \( t_1 \) con vector director \( (1, f(t,x')) \) y así sucesivamente.

Esta sucesion de segmentos se conoce con el nombre de polnomial de Euler asociada a (P) y a la partición \( \Pi \).
Si definición formal es así:
\[
  p(t) =
  \begin{cases}
    x' \qquad si \qquad t = t_0 \\
  p(t_i)+(t-t_i)f(t_i,p(t_i)) \qquad si \qquad t \in (t_i, t_{i+1}], i \in \{0,...,m-1\}}
  \end{cases}
\]

\subsection{Nota}
1. Para que la definicion de \( () \)

\subsection{Propiedades}
1. p es continua en I (si es que esta bien definida)

2. p es derivalbe en I, salvo a lo sumo en los putnos \( t_0, ..., t_{m} \).
En estos puntos existen las derivadas laterales 


\section{Definición de soluciones \( \varepsilon \) - aproximadas}

Sea \( ||.||_{ \mathbb{R}^{n} } \) una norma en \( \mathbb{R}^{n} \) (n>1).
Sea \( f : D \subset \mathbb{R}^{ } \times \mathbb{R}^{n} \to \mathbb{R}^{n} \)
Sea \( \varepsilon >0 \).

Una solución \( \varepsilon \)-aproximada de la ecuación diferencial (E) \( x'=f(t,x) \) respecto de \( ||.||_{ \mathbb{R}^{n} } \) es una función \( \varphi : I \to \mathbb{R}^{n} \), donde I es intervalo de \( \mathbb{R}^{ } \) que verifica:

\begin{enumerate}
  \item (E1) \( Graf(\varphi) \subset D\)
  \item (E2) \( \varphi \) es continua en I.
  \item \( \varphi \) es derivable en I, salvo quizá en un número finito de puntos \( t_1, ,..., t_m \). En estos puntos , existen las derivadas laterales y \( \varphi' \)es continua en los subintervalos determinados por estos punots.
  \item \( ||\varphi'(t)-f(t,\varphi(t))||_{ \mathbb{R}^{n} } \leq \varepsilon \qquad \forall t \in I - \{t_1, ... , t_m\}\)
\end{enumerate}

Si además, \( \varphi(t_0)=x^0 \) con \( (t_0,x^0) \in D \) entonces decimos que \( \varphi: I \to \mathbb{R}^{n} \) es solución \( \varepsilon \)-aproximada respecto de la norma de \( \mathbb{R}^{n} \) del problema \( (P)
\begin{cases}
  x' = f(t,x) \\
  x(t_0) = x^0
\end{cases} \)

\section{Teorema}

Teorema de existencia de soluciones \( \varepsilon \)-aproximadas (versión lateral derecha).

Sea \( ||.||_{ \mathbb{R}^{n} } \) una norma en \( \mathbb{R}^{n} \) (n>1).

Consideramos el problema de Cauchy \( (P)
\begin{cases}
  x' = f(t,x) \\
  x(t_0) = x^0
\end{cases} \)
donde \( f : D \subset \mathbb{R}^{ } \times \mathbb{R}^{n}\to \mathbb{R}^{n} \) y \( (t_0,x^0) \in D \) 
